% Options for packages loaded elsewhere
\PassOptionsToPackage{unicode}{hyperref}
\PassOptionsToPackage{hyphens}{url}
\PassOptionsToPackage{dvipsnames,svgnames,x11names}{xcolor}
%
\documentclass[
  letterpaper,
  DIV=11,
  numbers=noendperiod]{scrreprt}

\usepackage{amsmath,amssymb}
\usepackage{iftex}
\ifPDFTeX
  \usepackage[T1]{fontenc}
  \usepackage[utf8]{inputenc}
  \usepackage{textcomp} % provide euro and other symbols
\else % if luatex or xetex
  \usepackage{unicode-math}
  \defaultfontfeatures{Scale=MatchLowercase}
  \defaultfontfeatures[\rmfamily]{Ligatures=TeX,Scale=1}
\fi
\usepackage{lmodern}
\ifPDFTeX\else  
    % xetex/luatex font selection
\fi
% Use upquote if available, for straight quotes in verbatim environments
\IfFileExists{upquote.sty}{\usepackage{upquote}}{}
\IfFileExists{microtype.sty}{% use microtype if available
  \usepackage[]{microtype}
  \UseMicrotypeSet[protrusion]{basicmath} % disable protrusion for tt fonts
}{}
\makeatletter
\@ifundefined{KOMAClassName}{% if non-KOMA class
  \IfFileExists{parskip.sty}{%
    \usepackage{parskip}
  }{% else
    \setlength{\parindent}{0pt}
    \setlength{\parskip}{6pt plus 2pt minus 1pt}}
}{% if KOMA class
  \KOMAoptions{parskip=half}}
\makeatother
\usepackage{xcolor}
\setlength{\emergencystretch}{3em} % prevent overfull lines
\setcounter{secnumdepth}{5}
% Make \paragraph and \subparagraph free-standing
\makeatletter
\ifx\paragraph\undefined\else
  \let\oldparagraph\paragraph
  \renewcommand{\paragraph}{
    \@ifstar
      \xxxParagraphStar
      \xxxParagraphNoStar
  }
  \newcommand{\xxxParagraphStar}[1]{\oldparagraph*{#1}\mbox{}}
  \newcommand{\xxxParagraphNoStar}[1]{\oldparagraph{#1}\mbox{}}
\fi
\ifx\subparagraph\undefined\else
  \let\oldsubparagraph\subparagraph
  \renewcommand{\subparagraph}{
    \@ifstar
      \xxxSubParagraphStar
      \xxxSubParagraphNoStar
  }
  \newcommand{\xxxSubParagraphStar}[1]{\oldsubparagraph*{#1}\mbox{}}
  \newcommand{\xxxSubParagraphNoStar}[1]{\oldsubparagraph{#1}\mbox{}}
\fi
\makeatother


\providecommand{\tightlist}{%
  \setlength{\itemsep}{0pt}\setlength{\parskip}{0pt}}\usepackage{longtable,booktabs,array}
\usepackage{calc} % for calculating minipage widths
% Correct order of tables after \paragraph or \subparagraph
\usepackage{etoolbox}
\makeatletter
\patchcmd\longtable{\par}{\if@noskipsec\mbox{}\fi\par}{}{}
\makeatother
% Allow footnotes in longtable head/foot
\IfFileExists{footnotehyper.sty}{\usepackage{footnotehyper}}{\usepackage{footnote}}
\makesavenoteenv{longtable}
\usepackage{graphicx}
\makeatletter
\def\maxwidth{\ifdim\Gin@nat@width>\linewidth\linewidth\else\Gin@nat@width\fi}
\def\maxheight{\ifdim\Gin@nat@height>\textheight\textheight\else\Gin@nat@height\fi}
\makeatother
% Scale images if necessary, so that they will not overflow the page
% margins by default, and it is still possible to overwrite the defaults
% using explicit options in \includegraphics[width, height, ...]{}
\setkeys{Gin}{width=\maxwidth,height=\maxheight,keepaspectratio}
% Set default figure placement to htbp
\makeatletter
\def\fps@figure{htbp}
\makeatother

\KOMAoption{captions}{tableheading,figureheading}
\makeatletter
\@ifpackageloaded{tcolorbox}{}{\usepackage[skins,breakable]{tcolorbox}}
\@ifpackageloaded{fontawesome5}{}{\usepackage{fontawesome5}}
\definecolor{quarto-callout-color}{HTML}{909090}
\definecolor{quarto-callout-note-color}{HTML}{0758E5}
\definecolor{quarto-callout-important-color}{HTML}{CC1914}
\definecolor{quarto-callout-warning-color}{HTML}{EB9113}
\definecolor{quarto-callout-tip-color}{HTML}{00A047}
\definecolor{quarto-callout-caution-color}{HTML}{FC5300}
\definecolor{quarto-callout-color-frame}{HTML}{acacac}
\definecolor{quarto-callout-note-color-frame}{HTML}{4582ec}
\definecolor{quarto-callout-important-color-frame}{HTML}{d9534f}
\definecolor{quarto-callout-warning-color-frame}{HTML}{f0ad4e}
\definecolor{quarto-callout-tip-color-frame}{HTML}{02b875}
\definecolor{quarto-callout-caution-color-frame}{HTML}{fd7e14}
\makeatother
\makeatletter
\@ifpackageloaded{bookmark}{}{\usepackage{bookmark}}
\makeatother
\makeatletter
\@ifpackageloaded{caption}{}{\usepackage{caption}}
\AtBeginDocument{%
\ifdefined\contentsname
  \renewcommand*\contentsname{Table of contents}
\else
  \newcommand\contentsname{Table of contents}
\fi
\ifdefined\listfigurename
  \renewcommand*\listfigurename{List of Figures}
\else
  \newcommand\listfigurename{List of Figures}
\fi
\ifdefined\listtablename
  \renewcommand*\listtablename{List of Tables}
\else
  \newcommand\listtablename{List of Tables}
\fi
\ifdefined\figurename
  \renewcommand*\figurename{Figure}
\else
  \newcommand\figurename{Figure}
\fi
\ifdefined\tablename
  \renewcommand*\tablename{Table}
\else
  \newcommand\tablename{Table}
\fi
}
\@ifpackageloaded{float}{}{\usepackage{float}}
\floatstyle{ruled}
\@ifundefined{c@chapter}{\newfloat{codelisting}{h}{lop}}{\newfloat{codelisting}{h}{lop}[chapter]}
\floatname{codelisting}{Listing}
\newcommand*\listoflistings{\listof{codelisting}{List of Listings}}
\makeatother
\makeatletter
\makeatother
\makeatletter
\@ifpackageloaded{caption}{}{\usepackage{caption}}
\@ifpackageloaded{subcaption}{}{\usepackage{subcaption}}
\makeatother

\ifLuaTeX
  \usepackage{selnolig}  % disable illegal ligatures
\fi
\usepackage{bookmark}

\IfFileExists{xurl.sty}{\usepackage{xurl}}{} % add URL line breaks if available
\urlstyle{same} % disable monospaced font for URLs
\hypersetup{
  pdftitle={Template},
  pdfauthor={TWU Online},
  colorlinks=true,
  linkcolor={blue},
  filecolor={Maroon},
  citecolor={Blue},
  urlcolor={Blue},
  pdfcreator={LaTeX via pandoc}}


\title{Template}
\author{TWU Online}
\date{Oct 1, 2024}

\begin{document}
\maketitle

\renewcommand*\contentsname{Table of contents}
{
\hypersetup{linkcolor=}
\setcounter{tocdepth}{2}
\tableofcontents
}

\bookmarksetup{startatroot}

\chapter*{Welcome}\label{welcome}
\addcontentsline{toc}{chapter}{Welcome}

\markboth{Welcome}{Welcome}

This is the course book for \textbf{{[}Course Name{]}}. This book is
divided into thematic units of study to help you engage with the
materials. The course resources and learning activities are designed not
only to help prepare you for the course assessments, but also to give
you opportunities to practice various skills.

\begin{tcolorbox}[enhanced jigsaw, bottomrule=.15mm, colframe=quarto-callout-note-color-frame, arc=.35mm, rightrule=.15mm, opacityback=0, toprule=.15mm, breakable, leftrule=.75mm, left=2mm, colback=white]
\begin{minipage}[t]{5.5mm}
\textcolor{quarto-callout-note-color}{\faInfo}
\end{minipage}%
\begin{minipage}[t]{\textwidth - 5.5mm}

Please read the full course syllabus located on the Course Home page in
Moodle. It includes key information about the course schedule,
assignments, and policies.

\end{minipage}%
\end{tcolorbox}

\section*{Course Activities}\label{course-activities}
\addcontentsline{toc}{section}{Course Activities}

\markright{Course Activities}

Below is some key information on features you may see throughout the
course.

\begin{tcolorbox}[enhanced jigsaw, colbacktitle=quarto-callout-note-color!10!white, coltitle=black, colframe=quarto-callout-note-color-frame, rightrule=.15mm, bottomrule=.15mm, title={Learning Activity}, bottomtitle=1mm, left=2mm, toprule=.15mm, opacitybacktitle=0.6, arc=.35mm, opacityback=0, toptitle=1mm, breakable, leftrule=.75mm, titlerule=0mm, colback=white]

This box will prompt you to engage in course concepts by:

\begin{itemize}
\tightlist
\item
  Viewing resources and reflecting on your experience and/or learning.
\item
  Checking your understanding to make sure you are ready for what
  follows. Ways to check your learning might include self-check quizzes
  or questions for discussion.
\end{itemize}

\begin{tcolorbox}[enhanced jigsaw, bottomrule=.15mm, colframe=quarto-callout-note-color-frame, arc=.35mm, rightrule=.15mm, opacityback=0, toprule=.15mm, breakable, leftrule=.75mm, left=2mm, colback=white]

Working through course activities will help you to meet the learning
outcomes and successfully complete your assessments.

\end{tcolorbox}

\end{tcolorbox}

Below is an accordion.

\begin{tcolorbox}[enhanced jigsaw, bottomrule=.15mm, colframe=quarto-callout-note-color-frame, arc=.35mm, rightrule=.15mm, opacityback=0, toprule=.15mm, breakable, leftrule=.75mm, left=2mm, colback=white]

\vspace{-3mm}\textbf{This is an accordion. Click/tap this banner to show/hide the content.}\vspace{3mm}

An accordion may contain extra content such as worked examples or sample
answers.

\end{tcolorbox}

\bookmarksetup{startatroot}

\chapter{A title for Unit 1}\label{a-title-for-unit-1}

Welcome to {[}COURSE{]}

\section*{Topics}\label{topics}
\addcontentsline{toc}{section}{Topics}

\markright{Topics}

This unit is divided into the following topics: 1. 2. 3.

\subsection*{Unit Learning Outcomes}\label{unit-learning-outcomes}
\addcontentsline{toc}{subsection}{Unit Learning Outcomes}

When you have completed this unit, you will be able to:

\begin{itemize}
\tightlist
\item
  Describe\ldots{}
\item
  Contrast\ldots{}
\item
  Analyze\ldots{}
\item
  Determine\ldots{}
\item
  Create\ldots{}
\end{itemize}

\subsection*{Learning Activities}\label{learning-activities}
\addcontentsline{toc}{subsection}{Learning Activities}

Here is a list of learning activities that will benefit you in
completing this unit. You may find it useful for planning your work.

\begin{enumerate}
\def\labelenumi{\arabic{enumi}.}
\tightlist
\item
  Read\ldots{}
\item
  Watch\ldots{}
\item
  Explore\ldots{}
\item
  Complete the ungraded quiz.
\end{enumerate}

\begin{tcolorbox}[enhanced jigsaw, bottomrule=.15mm, colframe=quarto-callout-note-color-frame, arc=.35mm, rightrule=.15mm, opacityback=0, toprule=.15mm, breakable, leftrule=.75mm, left=2mm, colback=white]
\begin{minipage}[t]{5.5mm}
\textcolor{quarto-callout-note-color}{\faInfo}
\end{minipage}%
\begin{minipage}[t]{\textwidth - 5.5mm}

Working through course activities will help you to meet the learning
outcomes and successfully complete your assessments.

\end{minipage}%
\end{tcolorbox}

\subsection*{Assessment}\label{assessment}
\addcontentsline{toc}{subsection}{Assessment}

Please see the Assessment section in Moodle for assignment details.

\subsection*{Resources}\label{resources}
\addcontentsline{toc}{subsection}{Resources}

Here are the resources you will need to complete this unit.

\begin{itemize}
\tightlist
\item
  (Textbook)
\item
  Other online resources will be provided in the unit.
\end{itemize}

\section{Title for Topic 1}\label{title-for-topic-1}

We begin Unit 1\ldots{}

\emph{(add content)}

\subsection*{Activity: Title (e.g.~Read, Reflect and
View)}\label{activity-title-e.g.-read-reflect-and-view}
\addcontentsline{toc}{subsection}{Activity: Title (e.g.~Read, Reflect
and View)}

\begin{tcolorbox}[enhanced jigsaw, colbacktitle=quarto-callout-note-color!10!white, coltitle=black, colframe=quarto-callout-note-color-frame, rightrule=.15mm, bottomrule=.15mm, title={Learning Activity}, bottomtitle=1mm, left=2mm, toprule=.15mm, opacitybacktitle=0.6, arc=.35mm, opacityback=0, toptitle=1mm, breakable, leftrule=.75mm, titlerule=0mm, colback=white]

\emph{(add content)}

View the following resources about \ldots{}

Next, watch the following videos that illustrate\ldots{}

\textbf{Questions to Consider}

After completing the activities above, consider the following questions:

\begin{itemize}
\tightlist
\item
  This is a question?
\end{itemize}

\end{tcolorbox}

\section{Title for Topic 2}\label{title-for-topic-2}

\emph{(add content)}

\ldots{}

\subsection*{Activity: Title}\label{activity-title}
\addcontentsline{toc}{subsection}{Activity: Title}

\begin{tcolorbox}[enhanced jigsaw, colbacktitle=quarto-callout-note-color!10!white, coltitle=black, colframe=quarto-callout-note-color-frame, rightrule=.15mm, bottomrule=.15mm, title={Learning Activity}, bottomtitle=1mm, left=2mm, toprule=.15mm, opacitybacktitle=0.6, arc=.35mm, opacityback=0, toptitle=1mm, breakable, leftrule=.75mm, titlerule=0mm, colback=white]

e.g.~Case study

\emph{(add content)}

Note that you may be asked to review this case or similar cases in your
class discussion groups. You may want to prepare by relating the case to
your readings. Specifically, identify the ethical issues and terms to
help explain the case.*

\emph{{[}Note for Facilitator's Guide: Use the case study above as a
class or group discussion prompt. Remind students to complete this
activity before the class session.{]}}

\end{tcolorbox}

\section{Title for Topic 3}\label{title-for-topic-3}

\emph{(add content)}

\ldots{}

\subsection*{Activity: Title}\label{activity-title-1}
\addcontentsline{toc}{subsection}{Activity: Title}

\begin{tcolorbox}[enhanced jigsaw, colbacktitle=quarto-callout-note-color!10!white, coltitle=black, colframe=quarto-callout-note-color-frame, rightrule=.15mm, bottomrule=.15mm, title={Learning Activity}, bottomtitle=1mm, left=2mm, toprule=.15mm, opacitybacktitle=0.6, arc=.35mm, opacityback=0, toptitle=1mm, breakable, leftrule=.75mm, titlerule=0mm, colback=white]

\emph{(See
\href{https://multi-access.twu.ca/assessment/assessment-ideas}{Assessment
ideas} for other ways to engage students in the topics.)}

\end{tcolorbox}

\subsection*{Activity: Key Terms Quiz
(ungraded)}\label{activity-key-terms-quiz-ungraded}
\addcontentsline{toc}{subsection}{Activity: Key Terms Quiz (ungraded)}

\begin{tcolorbox}[enhanced jigsaw, colbacktitle=quarto-callout-note-color!10!white, coltitle=black, colframe=quarto-callout-note-color-frame, rightrule=.15mm, bottomrule=.15mm, title={Learning Activity}, bottomtitle=1mm, left=2mm, toprule=.15mm, opacitybacktitle=0.6, arc=.35mm, opacityback=0, toptitle=1mm, breakable, leftrule=.75mm, titlerule=0mm, colback=white]

To review the concepts from this unit, take the following quiz.~ This
activity is ungraded and is designed to help prepare you for the
assessments in this course.

Match the following terms to their correct definition.

\begin{enumerate}
\def\labelenumi{\arabic{enumi}.}
\tightlist
\item
  \textbf{Term}-- definition\ldots{}
\item
  \textbf{Term}-- definition\ldots{}
\item
  \textbf{Term}-- definition\ldots{}
\end{enumerate}

\end{tcolorbox}

\section*{Unit Summary}\label{unit-summary}
\addcontentsline{toc}{section}{Unit Summary}

\markright{Unit Summary}

In this first unit, you have had the opportunity to learn about\ldots{}

\emph{(add content)}

\begin{tcolorbox}[enhanced jigsaw, colbacktitle=quarto-callout-note-color!10!white, coltitle=black, colframe=quarto-callout-note-color-frame, rightrule=.15mm, bottomrule=.15mm, title={Checking Your Learning}, bottomtitle=1mm, left=2mm, toprule=.15mm, opacitybacktitle=0.6, arc=.35mm, opacityback=0, toptitle=1mm, breakable, leftrule=.75mm, titlerule=0mm, colback=white]

Before you move on to the next unit, you may want to check that you are
able to:

\begin{itemize}
\tightlist
\item
  Describe\ldots{}
\item
  Contrast\ldots{}
\item
  Analyze\ldots{}
\item
  Determine\ldots{}
\item
  Create\ldots{}
\end{itemize}

\end{tcolorbox}

\bookmarksetup{startatroot}

\chapter{A title for Unit 2}\label{a-title-for-unit-2}

\bookmarksetup{startatroot}

\chapter{\# A title for Unit 3}\label{a-title-for-unit-3}

This is a book created from markdown and executable code.

\bookmarksetup{startatroot}

\chapter{\# A title for Unit 4}\label{a-title-for-unit-4}

This is a book created from markdown and executable code.

\bookmarksetup{startatroot}

\chapter{\# A title for Unit 5}\label{a-title-for-unit-5}

This is a book created from markdown and executable code.

\bookmarksetup{startatroot}

\chapter{\# A title for Unit 6}\label{a-title-for-unit-6}

This is a book created from markdown and executable code.




\end{document}
